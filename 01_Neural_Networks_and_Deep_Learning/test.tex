%! Author = user
%! Date = 3/9/2024

% Preamble
\documentclass{article}
\usepackage[utf8]{inputenc}
\usepackage{amsmath, amssymb, amsthm}

\title{LaTex for Students}
\author{Jun Ho Lee}
\date{March 2024}

% Document
\begin{document}

\newpage

\section{Week1}

Introduction to Deep Learning.

    \begin{itemize}
        \item[--]{Week1. Introduction}
        \item[--]{Week2. Basics of Neural Network programming}
        \item[--]{Week3. One hidden layer Neural Networks}
        \item[--]{Week4. Deep Neural Networks}
    \end{itemize}

\subsection{What is a Neural Network?}

\subsection{Supervised Learning with Neural Networks}

\subsection{Why is Deep Learning taking off?}

\newpage

\section{Week2}

Basics of Neural Network Programming

\subsection{Neural Network Notations}

How do I write an equation in \LaTeX?\\

\begin{itemize}
    \item[-]{superscript (i) : $i^{th}$ training example}
    \item[-]{superscript (l) : $l^{th}$ layer}
    \item[-]{m: number of examples in the dataset}
    \item[-]{$n_x$: input size}
    \item[-]{$n_y$: output size}
\end{itemize}

    In 1902, Einstein created this equation: $E=mc^2$

    And Newton came up with this one: $\sum F=ma$

    \begin{equation}
        5+5=10
    \end{equation}

    \begin{equation}
        \begin{split}
            A & = \frac{5\pi r^2}{2} \\
            A & = \frac{1}{2} \pi r^2
        \end{split}
    \end{equation}

\end{document}